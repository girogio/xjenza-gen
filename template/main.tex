% FILENAME: main.tex

\input{./packages/xjenza-preamble.tex}

% Front matter
\title{\BLOCK{ if title is defined }\VAR{title}\BLOCK{ else }A very interesting scientific publication\BLOCK{ endif }}
\doi{TO-BE-ASSIGNED}
\articleType{Research Article}%Research Article, Review Article, Research Note, News Article...
\author{\BLOCK{ if authors is defined }\VAR{authors_string}\BLOCK{ else }J. Borg$^{1}$, J. Doe$^{2*}$ and J. Bloggs$^{2}$\BLOCK{ endif }}
\authorAnnotation{\BLOCK{ if affiliations is defined }\VAR{affiliations}\BLOCK{ else }$^{2}$Department of Physics, Harvard University, Cambridge MA, US\BLOCK{ endif }}
\correspondanceName{\BLOCK{ if corresponder is defined }\VAR{corresponder.name[0]}. \VAR{corresponder.surname}\BLOCK{ else }John Doe\BLOCK{ endif }}
\correspondanceMail{\BLOCK{ if corresponder is defined }\VAR{corresponder.email}\BLOCK{ else }john@doe.com\BLOCK{ endif }}

% This goes in the header, in case the full title is too long
% TODO: make this automatic
\shortTitle{A very interesting scientific publication}

\selfCitation{\BLOCK{ if authors is defined }\VAR{authors[0].surname}, \VAR{authors[0].name[0]} et al.\BLOCK{ else }Borg, J. et al.\BLOCK{ endif } (\VAR{year}).\newblock {\em Xjenza Online}, \pageref{firstpage}--\pageref{lastPage}.}

% Bibliography file
\addbibresource{references.bib}

% Insert additional packages here
\usepackage{lipsum}

%%%%% Do not edit %%%%%
\def\firstpage{1}
\newcounter{pagna}
\setcounter{pagna}{\firstpage}
\setcounter{page}{\firstpage}
%%%%%%%%%%%%%%%%%%%%%%%


\begin{document}
	\label{firstpage}

	\abstrac{\lipsum[1-3]}

	\maketitle

	\section{Introduction}
	\lipsum[4-6]

    \subsection{Citations}
    Here, we refer to \textcite{smit54,colu92}. We can also use parentheses to cite \parencite{phil99}.

    \section{Figures and Tables}
    In \cref{fig:mcsLogo}, we can see a the MCS logo, which occupies 30\% of the width of a column. Notice we use the \verb|\cref{}| command to get a clickable link.

    \begin{figure}[h]
    	\centering
    	\includegraphics[width=.3\linewidth]{packages/logo}
    	\caption{MCS Logo}
	    \label{fig:mcsLogo}
    \end{figure}

    If we have a large figure or table, and we want to break out of the two-column layout, occupying the whole page, we could instead make use a starred environment (\verb|\begin{figure*}...\end{figure*}| instead of \verb|\begin{figure}...\end{figure}|). For instance,
    \cref{tab:aSampleTable} does this.
    \begin{table*}[ht]
    	\centering
    	\renewcommand\arraystretch{1.3} % This adds more vertical spacing
    	\begin{tabular}{l|c p{3cm}}
    	    Column 1 & Column 2 & Column 3 \\
    	    \hline
    	    Left-aligned text & Centred text & A column no wider than 3cm \\
    	    x & y & z
    	\end{tabular}
    	\caption{A sample table}
	    \label{tab:aSampleTable}
    \end{table*}
    \LaTeX{} tables are not the easiest thing to work with, if you want you can use a spreadsheet software and then paste a table into a tool like \url{https://www.tablesgenerator.com/}, which will generate a table you can copy and paste into a \LaTeX{} document.

    % List all citations
	\nocite{*}
	\printbibliography

	% Balance columns on last page
	\balance

    % For header on first page
	\label{lastPage}
\end{document}
